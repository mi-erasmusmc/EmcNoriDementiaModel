% Options for packages loaded elsewhere
\PassOptionsToPackage{unicode}{hyperref}
\PassOptionsToPackage{hyphens}{url}
%
\documentclass[
]{article}
\usepackage{lmodern}
\usepackage{amssymb,amsmath}
\usepackage{ifxetex,ifluatex}
\ifnum 0\ifxetex 1\fi\ifluatex 1\fi=0 % if pdftex
  \usepackage[T1]{fontenc}
  \usepackage[utf8]{inputenc}
  \usepackage{textcomp} % provide euro and other symbols
\else % if luatex or xetex
  \usepackage{unicode-math}
  \defaultfontfeatures{Scale=MatchLowercase}
  \defaultfontfeatures[\rmfamily]{Ligatures=TeX,Scale=1}
\fi
% Use upquote if available, for straight quotes in verbatim environments
\IfFileExists{upquote.sty}{\usepackage{upquote}}{}
\IfFileExists{microtype.sty}{% use microtype if available
  \usepackage[]{microtype}
  \UseMicrotypeSet[protrusion]{basicmath} % disable protrusion for tt fonts
}{}
\makeatletter
\@ifundefined{KOMAClassName}{% if non-KOMA class
  \IfFileExists{parskip.sty}{%
    \usepackage{parskip}
  }{% else
    \setlength{\parindent}{0pt}
    \setlength{\parskip}{6pt plus 2pt minus 1pt}}
}{% if KOMA class
  \KOMAoptions{parskip=half}}
\makeatother
\usepackage{xcolor}
\IfFileExists{xurl.sty}{\usepackage{xurl}}{} % add URL line breaks if available
\IfFileExists{bookmark.sty}{\usepackage{bookmark}}{\usepackage{hyperref}}
\hypersetup{
  pdftitle={Using the measurement cohort covariate code},
  pdfauthor={Jenna M. Reps},
  hidelinks,
  pdfcreator={LaTeX via pandoc}}
\urlstyle{same} % disable monospaced font for URLs
\usepackage[margin=1in]{geometry}
\usepackage{color}
\usepackage{fancyvrb}
\newcommand{\VerbBar}{|}
\newcommand{\VERB}{\Verb[commandchars=\\\{\}]}
\DefineVerbatimEnvironment{Highlighting}{Verbatim}{commandchars=\\\{\}}
% Add ',fontsize=\small' for more characters per line
\usepackage{framed}
\definecolor{shadecolor}{RGB}{248,248,248}
\newenvironment{Shaded}{\begin{snugshade}}{\end{snugshade}}
\newcommand{\AlertTok}[1]{\textcolor[rgb]{0.94,0.16,0.16}{#1}}
\newcommand{\AnnotationTok}[1]{\textcolor[rgb]{0.56,0.35,0.01}{\textbf{\textit{#1}}}}
\newcommand{\AttributeTok}[1]{\textcolor[rgb]{0.77,0.63,0.00}{#1}}
\newcommand{\BaseNTok}[1]{\textcolor[rgb]{0.00,0.00,0.81}{#1}}
\newcommand{\BuiltInTok}[1]{#1}
\newcommand{\CharTok}[1]{\textcolor[rgb]{0.31,0.60,0.02}{#1}}
\newcommand{\CommentTok}[1]{\textcolor[rgb]{0.56,0.35,0.01}{\textit{#1}}}
\newcommand{\CommentVarTok}[1]{\textcolor[rgb]{0.56,0.35,0.01}{\textbf{\textit{#1}}}}
\newcommand{\ConstantTok}[1]{\textcolor[rgb]{0.00,0.00,0.00}{#1}}
\newcommand{\ControlFlowTok}[1]{\textcolor[rgb]{0.13,0.29,0.53}{\textbf{#1}}}
\newcommand{\DataTypeTok}[1]{\textcolor[rgb]{0.13,0.29,0.53}{#1}}
\newcommand{\DecValTok}[1]{\textcolor[rgb]{0.00,0.00,0.81}{#1}}
\newcommand{\DocumentationTok}[1]{\textcolor[rgb]{0.56,0.35,0.01}{\textbf{\textit{#1}}}}
\newcommand{\ErrorTok}[1]{\textcolor[rgb]{0.64,0.00,0.00}{\textbf{#1}}}
\newcommand{\ExtensionTok}[1]{#1}
\newcommand{\FloatTok}[1]{\textcolor[rgb]{0.00,0.00,0.81}{#1}}
\newcommand{\FunctionTok}[1]{\textcolor[rgb]{0.00,0.00,0.00}{#1}}
\newcommand{\ImportTok}[1]{#1}
\newcommand{\InformationTok}[1]{\textcolor[rgb]{0.56,0.35,0.01}{\textbf{\textit{#1}}}}
\newcommand{\KeywordTok}[1]{\textcolor[rgb]{0.13,0.29,0.53}{\textbf{#1}}}
\newcommand{\NormalTok}[1]{#1}
\newcommand{\OperatorTok}[1]{\textcolor[rgb]{0.81,0.36,0.00}{\textbf{#1}}}
\newcommand{\OtherTok}[1]{\textcolor[rgb]{0.56,0.35,0.01}{#1}}
\newcommand{\PreprocessorTok}[1]{\textcolor[rgb]{0.56,0.35,0.01}{\textit{#1}}}
\newcommand{\RegionMarkerTok}[1]{#1}
\newcommand{\SpecialCharTok}[1]{\textcolor[rgb]{0.00,0.00,0.00}{#1}}
\newcommand{\SpecialStringTok}[1]{\textcolor[rgb]{0.31,0.60,0.02}{#1}}
\newcommand{\StringTok}[1]{\textcolor[rgb]{0.31,0.60,0.02}{#1}}
\newcommand{\VariableTok}[1]{\textcolor[rgb]{0.00,0.00,0.00}{#1}}
\newcommand{\VerbatimStringTok}[1]{\textcolor[rgb]{0.31,0.60,0.02}{#1}}
\newcommand{\WarningTok}[1]{\textcolor[rgb]{0.56,0.35,0.01}{\textbf{\textit{#1}}}}
\usepackage{longtable,booktabs}
% Correct order of tables after \paragraph or \subparagraph
\usepackage{etoolbox}
\makeatletter
\patchcmd\longtable{\par}{\if@noskipsec\mbox{}\fi\par}{}{}
\makeatother
% Allow footnotes in longtable head/foot
\IfFileExists{footnotehyper.sty}{\usepackage{footnotehyper}}{\usepackage{footnote}}
\makesavenoteenv{longtable}
\usepackage{graphicx,grffile}
\makeatletter
\def\maxwidth{\ifdim\Gin@nat@width>\linewidth\linewidth\else\Gin@nat@width\fi}
\def\maxheight{\ifdim\Gin@nat@height>\textheight\textheight\else\Gin@nat@height\fi}
\makeatother
% Scale images if necessary, so that they will not overflow the page
% margins by default, and it is still possible to overwrite the defaults
% using explicit options in \includegraphics[width, height, ...]{}
\setkeys{Gin}{width=\maxwidth,height=\maxheight,keepaspectratio}
% Set default figure placement to htbp
\makeatletter
\def\fps@figure{htbp}
\makeatother
\setlength{\emergencystretch}{3em} % prevent overfull lines
\providecommand{\tightlist}{%
  \setlength{\itemsep}{0pt}\setlength{\parskip}{0pt}}
\setcounter{secnumdepth}{5}

\title{Using the measurement cohort covariate code}
\author{Jenna M. Reps}
\date{2020-10-13}

\begin{document}
\maketitle

{
\setcounter{tocdepth}{2}
\tableofcontents
}
\hypertarget{introduction}{%
\section{Introduction}\label{introduction}}

This vignette describes how one can use the function
`createMeasurementCohortCovariateSettings' to define measurement
covariates that also require being in or not being in a cohort during
the time period using the OMOP CDM. You will need:

\begin{enumerate}
\def\labelenumi{\arabic{enumi}.}
\tightlist
\item
  A concept set for the measurements (a vector of
  measurement\_concept\_ids)
\item
  A function to standardise the measurements (e.g., filters out unlikely
  values or converts units)
\item
  A cohort database schema, a cohort table and cohort definition id for
  the cohort of interest
\item
  How to aggregate multiple measurement values (e.g., use most recent to
  index, max value, min value or mean value)
\end{enumerate}

\hypertarget{createmeasurementcohortcovariatesettings}{%
\subsection{createMeasurementCohortCovariateSettings}\label{createmeasurementcohortcovariatesettings}}

This function contains the settings required to define the measurement
cohort covariate. For a measurement cohort covariate, the code will
check the measurement table in the OMOP CDM to find all rows where the
measurement\_concept\_id is in the specified measurement concept set and
will then restrict to patients in (or not in if `type' == out) the
`cohortDatabaseSchema'.'cohortTable' with the cohort\_definition\_id of
`cohortId' between the index date plus the `startDay' and the index date
plus the `endDay'. It will then check whether the measurement\_date
column calls between the index date plus the `startDay' and the index
date plus the `endDay'. The `scaleMap' will map the measurement values
to a uniform scale - this standardises the values. If there are multiple
measurements within the time period then the `aggregateMethod' method
with specify how to get a single value. The settings `ageInteraction'
and `lnAgeInteraction' enable the user to create age/ln(age) interaction
terms. The `lnValue' enables the user to use the natural logarithm of
the measurment value. Finally, the `analysisId' is used to create the
cohort covariateId as 1000*`measurementId' + `analysisId'.

\begin{longtable}[]{@{}ll@{}}
\caption{The inputs into the create function}\tabularnewline
\toprule
Input & Description\tabularnewline
\midrule
\endfirsthead
\toprule
Input & Description\tabularnewline
\midrule
\endhead
covariateName & The name of the covariate\tabularnewline
covariateId & The id of the covariate - generally
measurementId*1000+analysisId\tabularnewline
cohortDatabaseSchema & The database schema with the cohort used to
create a covariate\tabularnewline
cohortTable & The table with the cohort used to create a
covariate\tabularnewline
cohortId & The cohort definition id for the cohort used to create a
covariate\tabularnewline
conseptSet & A vector of concept\_ids corresponding to the
measurement\tabularnewline
type & in or out - in means the patients with a measurement must be in
the cohort of interest during the start and end date and out means the
patients with a measurement must not be in the cohort of interest during
the start and end date\tabularnewline
startDay & How many days prior to index to see whether the measurement
occurs after\tabularnewline
endDay & How many days relative to index to see whether the mesurement
occurs before\tabularnewline
scaleMap & A function that takes the covariate Amdromeda table as input
and processes it - can include filtering invalid values or mapping based
on unit\_concept\_id values\tabularnewline
aggregateMethod & How to pick a measurement value when there are more
than 1 during the start and end dates - can be min/max/recent (closest
to index)/mean\tabularnewline
imputationValue & A value to use if a person has no measurement during
the start and end dates\tabularnewline
ageInteraction & Include interaction with age\tabularnewline
lnAgeInteraction & Include interaction with ln(age)\tabularnewline
lnValue & Whether to us the natural log of the measurement
value\tabularnewline
analysisId & The analysis id for the covariate\tabularnewline
\bottomrule
\end{longtable}

\hypertarget{example}{%
\subsection{Example}\label{example}}

Assuming the concept set c(3004249, 3009395, 3018586, 3028737, 3035856,
4152194, 4153323, 4161413, 4197167, 4217013, 4232915, 4248525, 4292062,
21492239, 37396683, 44789315, 44806887, 45769778) corresponds to
`Systolic blood pressure'. The cohort in `your database schema'.'cohort'
with cohort\_definition\_id of 123 corresponds to periods of time where
a patient is given an anti-hypertensive drug.

We create a function to map the covariate object (this contains the
measurementConceptId, unitConceptId, rawValue and valueAsNumber columns)
to standardise the measurement values. As the systolic blood pressure
measurements often have no unit\_cocept\_id we cannot standarise based
mapping the units. Instead, We remove unfeasible values such as any
values less than 50 and greater than 250.

\begin{Shaded}
\begin{Highlighting}[]
\ControlFlowTok{function}\NormalTok{(x)\{ x =}\StringTok{ }\NormalTok{dplyr}\OperatorTok{::}\KeywordTok{filter}\NormalTok{(x, rawValue }\OperatorTok{>=}\StringTok{ }\DecValTok{50} \OperatorTok{&}\StringTok{ }\NormalTok{rawValue }\OperatorTok{<=}\StringTok{ }\DecValTok{250}\NormalTok{ ); }\KeywordTok{return}\NormalTok{(x)\}}
\end{Highlighting}
\end{Shaded}

We include all measurement values with occured within 1 year prior to
index and up to 60 days after, but use the value that occurred closest
to index.

To create a treated systolic blood pressure covariate (a blood pressure
measurement where the patient is in the anti-hypertensive cohort) using
a measurement cohort covariate run:

\begin{Shaded}
\begin{Highlighting}[]
\NormalTok{cohortCov1 <-}\StringTok{ }\KeywordTok{createCohortCovariateSettings}\NormalTok{(}\DataTypeTok{covariateName =} \StringTok{'Treated systolic blood pressure'}\NormalTok{,}
                                            \DataTypeTok{covariateId =} \DecValTok{1}\OperatorTok{*}\DecValTok{1000}\OperatorTok{+}\DecValTok{458}\NormalTok{,}
                                            \DataTypeTok{cohortDatabaseSchema =} \StringTok{'your database schema'}\NormalTok{,}
                                            \DataTypeTok{cohortTable =} \StringTok{'cohort'}\NormalTok{,}
                                            \DataTypeTok{cohortId =} \DecValTok{123}\NormalTok{,}
                                            \DataTypeTok{conseptSet =} \KeywordTok{c}\NormalTok{(}\DecValTok{3004249}\NormalTok{, }\DecValTok{3009395}\NormalTok{, }\DecValTok{3018586}\NormalTok{, }\DecValTok{3028737}\NormalTok{, }\DecValTok{3035856}\NormalTok{, }\DecValTok{4152194}\NormalTok{, }\DecValTok{4153323}\NormalTok{, }\DecValTok{4161413}\NormalTok{, }\DecValTok{4197167}\NormalTok{, }\DecValTok{4217013}\NormalTok{, }\DecValTok{4232915}\NormalTok{, }\DecValTok{4248525}\NormalTok{, }\DecValTok{4292062}\NormalTok{, }\DecValTok{21492239}\NormalTok{, }\DecValTok{37396683}\NormalTok{, }\DecValTok{44789315}\NormalTok{, }\DecValTok{44806887}\NormalTok{, }\DecValTok{45769778}\NormalTok{),}
                                            \DataTypeTok{type =} \StringTok{'in'}\NormalTok{,}
                                            \DataTypeTok{startDay=} \DecValTok{-365}\NormalTok{, }
                                            \DataTypeTok{endDay=}\DecValTok{60}\NormalTok{,}
                                            \DataTypeTok{scaleMap =} \ControlFlowTok{function}\NormalTok{(x)\{ x =}\StringTok{ }\NormalTok{dplyr}\OperatorTok{::}\KeywordTok{filter}\NormalTok{(x, rawValue }\OperatorTok{>=}\StringTok{ }\DecValTok{50} \OperatorTok{&}\StringTok{ }\NormalTok{rawValue }\OperatorTok{<=}\StringTok{ }\DecValTok{250}\NormalTok{ ); }\KeywordTok{return}\NormalTok{(x)\},}
                                            \DataTypeTok{aggregateMethod=} \StringTok{'recent'}\NormalTok{, }
                                            \DataTypeTok{ageInteraction =} \OtherTok{FALSE}\NormalTok{,}
                                            \DataTypeTok{lnAgeInteraction =} \OtherTok{FALSE}\NormalTok{,}
                                            \DataTypeTok{lnValue =} \OtherTok{FALSE}\NormalTok{,}
                                            \DataTypeTok{analysisId =} \DecValTok{458}\NormalTok{)}
\end{Highlighting}
\end{Shaded}

To create an untreated systolic blood pressure covariate (a blood
pressure measurement where the patient is not in the anti-hypertensive
cohort) using a measurement cohort covariate run:

\begin{Shaded}
\begin{Highlighting}[]
\NormalTok{cohortCov1 <-}\StringTok{ }\KeywordTok{createCohortCovariateSettings}\NormalTok{(}\DataTypeTok{covariateName =} \StringTok{'Untreated systolic blood pressure'}\NormalTok{,}
                                            \DataTypeTok{covariateId =} \DecValTok{2}\OperatorTok{*}\DecValTok{1000}\OperatorTok{+}\DecValTok{458}\NormalTok{,}
                                            \DataTypeTok{cohortDatabaseSchema =} \StringTok{'your database schema'}\NormalTok{,}
                                            \DataTypeTok{cohortTable =} \StringTok{'cohort'}\NormalTok{,}
                                            \DataTypeTok{cohortId =} \DecValTok{123}\NormalTok{,}
                                            \DataTypeTok{conseptSet =} \KeywordTok{c}\NormalTok{(}\DecValTok{3004249}\NormalTok{, }\DecValTok{3009395}\NormalTok{, }\DecValTok{3018586}\NormalTok{, }\DecValTok{3028737}\NormalTok{, }\DecValTok{3035856}\NormalTok{, }\DecValTok{4152194}\NormalTok{, }\DecValTok{4153323}\NormalTok{, }\DecValTok{4161413}\NormalTok{, }\DecValTok{4197167}\NormalTok{, }\DecValTok{4217013}\NormalTok{, }\DecValTok{4232915}\NormalTok{, }\DecValTok{4248525}\NormalTok{, }\DecValTok{4292062}\NormalTok{, }\DecValTok{21492239}\NormalTok{, }\DecValTok{37396683}\NormalTok{, }\DecValTok{44789315}\NormalTok{, }\DecValTok{44806887}\NormalTok{, }\DecValTok{45769778}\NormalTok{),}
                                            \DataTypeTok{type =} \StringTok{'out'}\NormalTok{,}
                                            \DataTypeTok{startDay=} \DecValTok{-365}\NormalTok{, }
                                            \DataTypeTok{endDay=}\DecValTok{60}\NormalTok{,}
                                            \DataTypeTok{scaleMap =} \ControlFlowTok{function}\NormalTok{(x)\{ x =}\StringTok{ }\NormalTok{dplyr}\OperatorTok{::}\KeywordTok{filter}\NormalTok{(x, rawValue }\OperatorTok{>=}\StringTok{ }\DecValTok{50} \OperatorTok{&}\StringTok{ }\NormalTok{rawValue }\OperatorTok{<=}\StringTok{ }\DecValTok{250}\NormalTok{ ); }\KeywordTok{return}\NormalTok{(x)\},}
                                            \DataTypeTok{aggregateMethod=} \StringTok{'recent'}\NormalTok{, }
                                            \DataTypeTok{ageInteraction =} \OtherTok{FALSE}\NormalTok{,}
                                            \DataTypeTok{lnAgeInteraction =} \OtherTok{FALSE}\NormalTok{,}
                                            \DataTypeTok{lnValue =} \OtherTok{FALSE}\NormalTok{,}
                                            \DataTypeTok{analysisId =} \DecValTok{458}\NormalTok{)}
\end{Highlighting}
\end{Shaded}

You can use the ageInteraction, lnAgeInteraction and lnValue to do log
mapping or include age interaction terms.

To include all the above as covariates, combine them into a list:

\begin{Shaded}
\begin{Highlighting}[]
\NormalTok{cohortCov <-}\StringTok{ }\KeywordTok{list}\NormalTok{(cohortCov1,cohortCov2)}
\end{Highlighting}
\end{Shaded}

\end{document}
