% Options for packages loaded elsewhere
\PassOptionsToPackage{unicode}{hyperref}
\PassOptionsToPackage{hyphens}{url}
%
\documentclass[
]{article}
\usepackage{lmodern}
\usepackage{amssymb,amsmath}
\usepackage{ifxetex,ifluatex}
\ifnum 0\ifxetex 1\fi\ifluatex 1\fi=0 % if pdftex
  \usepackage[T1]{fontenc}
  \usepackage[utf8]{inputenc}
  \usepackage{textcomp} % provide euro and other symbols
\else % if luatex or xetex
  \usepackage{unicode-math}
  \defaultfontfeatures{Scale=MatchLowercase}
  \defaultfontfeatures[\rmfamily]{Ligatures=TeX,Scale=1}
\fi
% Use upquote if available, for straight quotes in verbatim environments
\IfFileExists{upquote.sty}{\usepackage{upquote}}{}
\IfFileExists{microtype.sty}{% use microtype if available
  \usepackage[]{microtype}
  \UseMicrotypeSet[protrusion]{basicmath} % disable protrusion for tt fonts
}{}
\makeatletter
\@ifundefined{KOMAClassName}{% if non-KOMA class
  \IfFileExists{parskip.sty}{%
    \usepackage{parskip}
  }{% else
    \setlength{\parindent}{0pt}
    \setlength{\parskip}{6pt plus 2pt minus 1pt}}
}{% if KOMA class
  \KOMAoptions{parskip=half}}
\makeatother
\usepackage{xcolor}
\IfFileExists{xurl.sty}{\usepackage{xurl}}{} % add URL line breaks if available
\IfFileExists{bookmark.sty}{\usepackage{bookmark}}{\usepackage{hyperref}}
\hypersetup{
  pdftitle={Populating the study package},
  pdfauthor={Jenna M. Reps},
  hidelinks,
  pdfcreator={LaTeX via pandoc}}
\urlstyle{same} % disable monospaced font for URLs
\usepackage[margin=1in]{geometry}
\usepackage{color}
\usepackage{fancyvrb}
\newcommand{\VerbBar}{|}
\newcommand{\VERB}{\Verb[commandchars=\\\{\}]}
\DefineVerbatimEnvironment{Highlighting}{Verbatim}{commandchars=\\\{\}}
% Add ',fontsize=\small' for more characters per line
\usepackage{framed}
\definecolor{shadecolor}{RGB}{248,248,248}
\newenvironment{Shaded}{\begin{snugshade}}{\end{snugshade}}
\newcommand{\AlertTok}[1]{\textcolor[rgb]{0.94,0.16,0.16}{#1}}
\newcommand{\AnnotationTok}[1]{\textcolor[rgb]{0.56,0.35,0.01}{\textbf{\textit{#1}}}}
\newcommand{\AttributeTok}[1]{\textcolor[rgb]{0.77,0.63,0.00}{#1}}
\newcommand{\BaseNTok}[1]{\textcolor[rgb]{0.00,0.00,0.81}{#1}}
\newcommand{\BuiltInTok}[1]{#1}
\newcommand{\CharTok}[1]{\textcolor[rgb]{0.31,0.60,0.02}{#1}}
\newcommand{\CommentTok}[1]{\textcolor[rgb]{0.56,0.35,0.01}{\textit{#1}}}
\newcommand{\CommentVarTok}[1]{\textcolor[rgb]{0.56,0.35,0.01}{\textbf{\textit{#1}}}}
\newcommand{\ConstantTok}[1]{\textcolor[rgb]{0.00,0.00,0.00}{#1}}
\newcommand{\ControlFlowTok}[1]{\textcolor[rgb]{0.13,0.29,0.53}{\textbf{#1}}}
\newcommand{\DataTypeTok}[1]{\textcolor[rgb]{0.13,0.29,0.53}{#1}}
\newcommand{\DecValTok}[1]{\textcolor[rgb]{0.00,0.00,0.81}{#1}}
\newcommand{\DocumentationTok}[1]{\textcolor[rgb]{0.56,0.35,0.01}{\textbf{\textit{#1}}}}
\newcommand{\ErrorTok}[1]{\textcolor[rgb]{0.64,0.00,0.00}{\textbf{#1}}}
\newcommand{\ExtensionTok}[1]{#1}
\newcommand{\FloatTok}[1]{\textcolor[rgb]{0.00,0.00,0.81}{#1}}
\newcommand{\FunctionTok}[1]{\textcolor[rgb]{0.00,0.00,0.00}{#1}}
\newcommand{\ImportTok}[1]{#1}
\newcommand{\InformationTok}[1]{\textcolor[rgb]{0.56,0.35,0.01}{\textbf{\textit{#1}}}}
\newcommand{\KeywordTok}[1]{\textcolor[rgb]{0.13,0.29,0.53}{\textbf{#1}}}
\newcommand{\NormalTok}[1]{#1}
\newcommand{\OperatorTok}[1]{\textcolor[rgb]{0.81,0.36,0.00}{\textbf{#1}}}
\newcommand{\OtherTok}[1]{\textcolor[rgb]{0.56,0.35,0.01}{#1}}
\newcommand{\PreprocessorTok}[1]{\textcolor[rgb]{0.56,0.35,0.01}{\textit{#1}}}
\newcommand{\RegionMarkerTok}[1]{#1}
\newcommand{\SpecialCharTok}[1]{\textcolor[rgb]{0.00,0.00,0.00}{#1}}
\newcommand{\SpecialStringTok}[1]{\textcolor[rgb]{0.31,0.60,0.02}{#1}}
\newcommand{\StringTok}[1]{\textcolor[rgb]{0.31,0.60,0.02}{#1}}
\newcommand{\VariableTok}[1]{\textcolor[rgb]{0.00,0.00,0.00}{#1}}
\newcommand{\VerbatimStringTok}[1]{\textcolor[rgb]{0.31,0.60,0.02}{#1}}
\newcommand{\WarningTok}[1]{\textcolor[rgb]{0.56,0.35,0.01}{\textbf{\textit{#1}}}}
\usepackage{longtable,booktabs}
% Correct order of tables after \paragraph or \subparagraph
\usepackage{etoolbox}
\makeatletter
\patchcmd\longtable{\par}{\if@noskipsec\mbox{}\fi\par}{}{}
\makeatother
% Allow footnotes in longtable head/foot
\IfFileExists{footnotehyper.sty}{\usepackage{footnotehyper}}{\usepackage{footnote}}
\makesavenoteenv{longtable}
\usepackage{graphicx,grffile}
\makeatletter
\def\maxwidth{\ifdim\Gin@nat@width>\linewidth\linewidth\else\Gin@nat@width\fi}
\def\maxheight{\ifdim\Gin@nat@height>\textheight\textheight\else\Gin@nat@height\fi}
\makeatother
% Scale images if necessary, so that they will not overflow the page
% margins by default, and it is still possible to overwrite the defaults
% using explicit options in \includegraphics[width, height, ...]{}
\setkeys{Gin}{width=\maxwidth,height=\maxheight,keepaspectratio}
% Set default figure placement to htbp
\makeatletter
\def\fps@figure{htbp}
\makeatother
\setlength{\emergencystretch}{3em} % prevent overfull lines
\providecommand{\tightlist}{%
  \setlength{\itemsep}{0pt}\setlength{\parskip}{0pt}}
\setcounter{secnumdepth}{5}

\title{Populating the study package}
\author{Jenna M. Reps}
\date{2020-10-14}

\begin{document}
\maketitle

{
\setcounter{tocdepth}{2}
\tableofcontents
}
\hypertarget{introduction}{%
\section{Introduction}\label{introduction}}

This vignette describes how one can populate the
SkeletonExistingModelStudy package with the target cohort, outcome
cohorts and model settings.

The first step is to make a copy of the `SkeletonExistingModelStudy' R
package template. You can do this by cloning the github package and then
copy and paste the directory to a location in your computer.

Open the Skeleton R project in R studio, this can be done by finding and
double clicking on the SkeletonExistingModelStudy.Rproj file in the
folder that you just copy and pasted. Once the package project is opened
in R studio there are 5 steps that must be followed:

\begin{enumerate}
\def\labelenumi{\arabic{enumi}.}
\tightlist
\item
  Source extras/populatePackage.Rto add functions needed to populate the
  package
\item
  Rename the package using: replaceName()
\item
  Add the target and outcome cohort using: populatePackageCohorts()
\item
  Specify the model using: populatePackageModels() - this can be run
  multiple times to add multiple models
\item
  Build the study package
\end{enumerate}

The package is now ready to run or share on GitHub.

\hypertarget{step-1-source-.extraspopulatepackage.r}{%
\subsection{Step 1: Source
`./extras/populatePackage.R'}\label{step-1-source-.extraspopulatepackage.r}}

The R script extras/populatePackage.R contains multiple functions that
can be used to populate the skeleton package.

To add the functions to your environment, make sure the package R
project is open in R studio and run:

\begin{Shaded}
\begin{Highlighting}[]
\KeywordTok{source}\NormalTok{(}\StringTok{'./extras/populatePackage.R'}\NormalTok{)}
\end{Highlighting}
\end{Shaded}

This will make the functions `replaceName()', `populatePackageCohorts()'
and `populatePackageModels()' available to use within your R session.

\hypertarget{step-2-renaming-the-package}{%
\subsection{Step 2: Renaming the
package}\label{step-2-renaming-the-package}}

The package is still currently called
`EmcNoriDementiaModel' but you need to rename all the
files with the name of your study. To make this easy I created a
function `replaceName()' that will automatically replace
`EmcNoriDementiaModel' with your new study name. Simply
run:

\begin{Shaded}
\begin{Highlighting}[]
\KeywordTok{replaceName}\NormalTok{(}\DataTypeTok{packageLocation =} \KeywordTok{getwd}\NormalTok{(), }
          \DataTypeTok{packageName =} \StringTok{'PooledCohortvalidation'}\NormalTok{)}
\end{Highlighting}
\end{Shaded}

When you open the r project it will set your working directory to the
package directory - so `packageLocation = getwd()' should work. If this
has not occurred, you may need to manually specify the location where
`SkeletonExistingModelStudy.Rproj' exists in the skeleton copy you made.
Set `packageName' to be the name you want to call your study. The
example above will replace `SkeletonExistingModelStudy' with
`PooledCohortvalidation' throughout the skeleton package.

It is a good practice to close the R project and reopen it once renamed.
The R project will now be renamed, in the example about it would be
called `PooledCohortvalidation.Rproj'.

\hypertarget{step-3-adding-the-target-and-outcome-cohorts}{%
\subsection{Step 3: Adding the Target and Outcome
cohorts}\label{step-3-adding-the-target-and-outcome-cohorts}}

The `populatePackageCohorts()' function requires users to specify:

\begin{itemize}
\tightlist
\item
  targetCohortIds - The ATLAS id/s for the target cohort/s
\item
  targetCohortNames - A string or vector of string with sharable name/s
  for the target cohort/s
\item
  outcomeIds - The ATLAS id/s for the outcome cohort/s
\item
  outcomeNames - A string or vector of string with sharable name/s for
  the outcome cohort/s
\item
  baseUrl - The url for the ATLAS webapi (this will be used to extract
  the ATLAS cohorts)
\end{itemize}

For example, to insert the target cohort:

\begin{itemize}
\tightlist
\item
  `Pooled cohort equation target - non-black male' - that has the atlas
  ID 18941
\item
  `Pooled cohort equation target - non-black female' - that has the
  atlas ID 18942
\item
  `Pooled cohort equation target - black female' - that has the atlas ID
  18943
\item
  `Pooled cohort equation target - black male' - that has the atlas ID
  18944
\end{itemize}

and the outcome cohorts:

\begin{itemize}
\tightlist
\item
  `Earliest of AMI ischemic stroke or cardiovascular death' - that has
  the atlas ID 18945
\item
  `Acute myocardial infarction events' - that has the atlas ID 18935
\item
  `Ischemic stroke events' - that has the atlas ID 18936
\end{itemize}

from the webapi at `\url{https://yourWebAPI}' run:

\begin{Shaded}
\begin{Highlighting}[]
\KeywordTok{populatePackageCohorts}\NormalTok{(}\DataTypeTok{targetCohortIds =} \KeywordTok{c}\NormalTok{(}\DecValTok{18941}\NormalTok{, }\DecValTok{18942}\NormalTok{, }\DecValTok{18943}\NormalTok{, }\DecValTok{18944}\NormalTok{),}
                       \DataTypeTok{targetCohortNames =} \KeywordTok{c}\NormalTok{(}\StringTok{'Pooled cohort equation target - non-black male'}\NormalTok{,}
                                             \StringTok{'Pooled cohort equation target - non-black female'}\NormalTok{,}
                                             \StringTok{'Pooled cohort equation target - black female'}\NormalTok{,}
                                             \StringTok{'Pooled cohort equation target - black male'}\NormalTok{),}
                       \DataTypeTok{outcomeIds =} \KeywordTok{c}\NormalTok{(}\DecValTok{18945}\NormalTok{, }\DecValTok{18935}\NormalTok{, }\DecValTok{18936}\NormalTok{),}
                       \DataTypeTok{outcomeNames =} \KeywordTok{c}\NormalTok{(}\StringTok{'Earliest of AMI ischemic stroke or cardiovascular death'}\NormalTok{,}
                                       \StringTok{'Acute myocardial infarction events'}\NormalTok{,}
                                       \StringTok{'Ischemic stroke events'}\NormalTok{),}
                       \DataTypeTok{baseUrl =} \StringTok{'https://yourWebAPI'}\NormalTok{)}
\end{Highlighting}
\end{Shaded}

This then inserts each cohort into `inst/cohorts' as a json with the
file name targetCohortNames or outcomeNames `.json' and inserts in
`inst/sql/sql\_server' as a sql with the file name targetCohortNames or
outcomeNames `.sql'.

It also creates a csv file in `inst/settings' named `CohortToCreate.csv'
that specifies all the target and outcome cohorts to generate when the
study is executed.

\hypertarget{step-4-specify-the-model}{%
\subsection{Step 4: Specify the model}\label{step-4-specify-the-model}}

The next step is to insert the model settings into the package. This can
be done by running the function `populatePackageModels()' per model to
incorporate.

There are 6 main inputs into this function:

\begin{longtable}[]{@{}ll@{}}
\caption{The inputs into the populatePackageModels
function}\tabularnewline
\toprule
Input & Description\tabularnewline
\midrule
\endfirsthead
\toprule
Input & Description\tabularnewline
\midrule
\endhead
modelname & The name of the model you are specifying\tabularnewline
standardCovariates & A list of settings for predictors that are
standardCovariates\tabularnewline
cohortCovariateSettings & A list of settings for predictors that are
cohort covariates\tabularnewline
measurementCovariateSettings & A list of settings for predictors that
are measurment covariates\tabularnewline
measurementCohortCovariateSettings & A list of settings for predictors
that are measurement cohort covariates\tabularnewline
finalMapping & The mapping from total score to predicted
risk\tabularnewline
\bottomrule
\end{longtable}

This 6 inputs specifiy the model name, the predictors and the final
model mapping.

\hypertarget{modelname}{%
\subsubsection{modelname}\label{modelname}}

The `modelname' input is use when saving the model settings. The main
model settings are saved in inst/settings as modelname\_model.csv. Each
row in modelname\_model.csv specifies a covariate and the corresponding
number of points (i.e., coefficient value) and settings required to
create the covariate using standardCovariates, createCohortCovariate,
createMeasurementCovariate, createMeasurementCohortCovariate or
createAgeCovariate.

\hypertarget{standardcovariates}{%
\subsubsection{standardCovariates}\label{standardcovariates}}

The `standardCovariates' input is used when you want to include standard
FeatureExtraction covariates. The input is a list of settings that
specify what standard covariates to use (when you execute the study
these will be used to configure
FeatureExtraction::createCovariateSettings() ) and the corresponding
points.

\begin{longtable}[]{@{}lll@{}}
\caption{The inputs into the standardCovariates function}\tabularnewline
\toprule
Input & Description & Example\tabularnewline
\midrule
\endfirsthead
\toprule
Input & Description & Example\tabularnewline
\midrule
\endhead
covariateId & A vector of the standard covariateIds to include &
c(12003, 13003, 8507001)\tabularnewline
covariateName & A vector of the corresponding names & c(``Age 60-64'',
``Age 65-70'', ``Male'')\tabularnewline
points & A vector of the corresponding points &
c(10,10,15)\tabularnewline
featureExtraction & A vector of the inputs needed in
FeatureExtraction::createCovariateSettings &
c(``useDemographicsAgeGroup'',``useDemographicsAgeGroup'',
``useDemographicsGender'')\tabularnewline
\bottomrule
\end{longtable}

The `covariateId' values must match the standard ids created by
`FeatureExtraction::createCovariateSettings'. These are often the
conceptId*1000+analysisId, where the analysisId can be found in
\url{https://github.com/OHDSI/FeatureExtraction/blob/master/inst/csv/PrespecAnalyses.csv}
. When the `populatePackageModels' function is executed with
`standardCovariates' set then the model settings
`inst/settings/modelname\_model.csv' is created or appended to with the
standard covariate settings. If your model does not require standard
covariates, then set `standardCovariates' to NULL.

\hypertarget{cohortcovariatesettings}{%
\subsubsection{cohortCovariateSettings}\label{cohortcovariatesettings}}

The `cohortCovariateSettings' is used when you want to include
covariates using cohort definitions. This can be useful for more complex
features such as a diabetes definition that requires a diabetes
condition record and a treatment record within 60 days. The
`cohortCovariateSettings' input is a list containing the following
entries:

\begin{longtable}[]{@{}lll@{}}
\caption{The inputs into the cohortCovariateSettings
function}\tabularnewline
\toprule
Input & Description & Example\tabularnewline
\midrule
\endfirsthead
\toprule
Input & Description & Example\tabularnewline
\midrule
\endhead
baseUrl & A baseUrl for the ATLAS webapi that contains the cohorts &
\url{https://yourWebAPI}\tabularnewline
atlasCovariateIds & A vector of the atlas cohort ids &
c(191324,184343)\tabularnewline
atlasCovariateNames & A vector of the atlas cohort names &
c(``Diabetes'',``Hypertension'')\tabularnewline
analysisIds & A vector of analysis ids for the cohort covariates &
c(456,456)\tabularnewline
startDays & Find cohort entries where the cohort\_start\_date occurs
after the the startDays relative to index &
c(-9999,-9999)\tabularnewline
endDays & Find cohort entries where the cohort\_start\_date occurs
before the the endsDays relative to index & c(-1,-1)\tabularnewline
points & A vector of the points & c(5,10)\tabularnewline
count & A vector specifying whether to use binary indicators or count
the number of cohort\_start\_dates per patients between index +
startDays and index + endsDays & c(F,F)\tabularnewline
ageInteraction & A vector specifying whether to use the age interaction
& c(F,F)\tabularnewline
lnAgeInteraction & A vector specifying whether to use the natural
logarithm log(age) interaction & c(F,F)\tabularnewline
\bottomrule
\end{longtable}

The `cohortCovariateSettings' saves settings required by
`createCohortCovariateSettings'. When the `populatePackageModels'
function is executed with `cohortCovariateSettings' set then the
specified ATLAS cohorts are downloaded into the package using the webapi
and the model settings `inst/settings/modelname\_model.csv' is created
or appended to with the cohort covariate settings. If you do not want to
include cohort covariates set `cohortCovariateSettings' to NULL.

\hypertarget{measurementcovariatesettings}{%
\subsubsection{measurementCovariateSettings}\label{measurementcovariatesettings}}

The `measurementCovariateSettings' is used when you want to include
measurement covariates. This can be useful for more complex features
such as total cholesterol. The `measurementCovariateSettings' input is a
list containing the following entries:

\begin{longtable}[]{@{}lll@{}}
\caption{The inputs into the measurementCovariateSettings
function}\tabularnewline
\toprule
Input & Description & Example\tabularnewline
\midrule
\endfirsthead
\toprule
Input & Description & Example\tabularnewline
\midrule
\endhead
names & A vector of the measurement covariate names & c(``HDL-C\_mgdL'',
``HDL-C\_mgdL'')\tabularnewline
conceptSets & A list containing vectors of the measurement concept sets
& list(c(2212449, 3003767, 3007070, 3007676, 3011884, 3013473, 3015204,
3016945, 3022449, 3023602), c(2212449, 3003767, 3007070, 3007676,
3011884, 3013473, 3015204, 3016945, 3022449, 3023602))\tabularnewline
startDays & Find measurement entries where the measurement\_date occurs
after the the startDays relative to index & c(-365, -365)\tabularnewline
endDays & Find measurement entries where the measurement\_date occurs
before the the endsDays relative to index & c(60,60)\tabularnewline
scaleMaps & A list of functions specifying how to standardise the
measurements & list(function(x)\{return(x)\}, function(x)\{return(x)\}
)\tabularnewline
points & A vector of the points & c(5,10)\tabularnewline
aggregateMethods & A vector specifying how to pick a single measurement
value per patient. Choose from recent (closest to index), min, max or
mean & c(``recent'',``recent'')\tabularnewline
imputationValues & A vector of values used to impute when a measurement
is missing for a patient & c(50,50)\tabularnewline
ageInteractions & A vector specifying whether to use the age interaction
& c(F,F)\tabularnewline
lnAgeInteractions & A vector specifying whether to use the natural
logarithm log(age) interaction & c(F,F)\tabularnewline
lnValues & A vector specifying whether to use the log measurement value
& c(F,F)\tabularnewline
measurementIds & A vector specifying the measurement ids &
c(1,2)\tabularnewline
analysisIds & A vector specifying the analysis ids &
c(458,458)\tabularnewline
\bottomrule
\end{longtable}

The `measurementCovariateSettings' saves settings required by
`createMeasurementCovariateSettings'. When the `populatePackageModels'
function is executed with `measurementCovariateSettings' set then the
model settings `inst/settings/modelname\_model.csv' is created or
appended to with the neasurement covariate settings. The mapping
functions for each measurement covariate are saved as rds files into the
`inst/settings' directory. If you do not want to include measurement
covariates set `measurementCovariateSettings' to NULL.

\hypertarget{measurementcohortcovariatesettings}{%
\subsubsection{measurementCohortCovariateSettings}\label{measurementcohortcovariatesettings}}

The `measurementCohortCovariateSettings' is used when you want to
include measurement covariates where patients are also in or not in some
specified cohort. This can be useful for more complex features such as
treated systolic blood pressure or untreated systolic blood pressure.
The `measurementCohortCovariateSettings' input is a list containing the
following entries:

\begin{longtable}[]{@{}lll@{}}
\caption{The inputs into the measurementCohortCovariateSettings
function}\tabularnewline
\toprule
Input & Description & Example\tabularnewline
\midrule
\endfirsthead
\toprule
Input & Description & Example\tabularnewline
\midrule
\endhead
names & A vector of the measurement covariate names & c(``HDL-C\_mgdL'',
``HDL-C\_mgdL'')\tabularnewline
atlasCovariateIds & A vector of the cohort ids &
c(18946,18946)\tabularnewline
types & A vector of ``in'' or ``out'' indicating whether to require
being in, or not in, the cohort. & c(``in'',``out'')\tabularnewline
conceptSets & A list containing vectors of the measurement concept sets
& list(c(2212449, 3003767, 3007070, 3007676, 3011884, 3013473, 3015204,
3016945, 3022449, 3023602), c(2212449, 3003767, 3007070, 3007676,
3011884, 3013473, 3015204, 3016945, 3022449, 3023602))\tabularnewline
startDays & Find measurement entries where the measurement\_date occurs
after the the startDays relative to index & c(-365, -365)\tabularnewline
endDays & Find measurement entries where the measurement\_date occurs
before the the endsDays relative to index & c(60,60)\tabularnewline
scaleMaps & A list of functions specifying how to standardise the
measurements & list(function(x)\{return(x)\}, function(x)\{return(x)\}
)\tabularnewline
points & A vector of the points & c(5,10)\tabularnewline
aggregateMethods & A vector specifying how to pick a single measurement
value per patient. Choose from recent (closest to index), min, max or
mean & c(``recent'',``recent'')\tabularnewline
ageInteractions & A vector specifying whether to use the age interaction
& c(F,F)\tabularnewline
lnAgeInteractions & A vector specifying whether to use the natural
logarithm log(age) interaction & c(F,F)\tabularnewline
lnValues & A vector specifying whether to use the log measurement value
& c(F,F)\tabularnewline
measurementIds & A vector specifying the measurement ids &
c(1,2)\tabularnewline
analysisIds & A vector specifying the analysis ids &
c(458,458)\tabularnewline
baseUrl & ATLAS webapi address & \url{https://yourWebAPI}\tabularnewline
\bottomrule
\end{longtable}

The `measurementCohortCovariateSettings' saves settings required by
`createMeasurementCohortCovariateSettings'. When the
`populatePackageModels' function is executed with
`measurementCohortCovariateSettings' set then the model settings
`inst/settings/modelname\_model.csv' is created or appended to with the
neasurement covariate settings and all specified ATLAS cohort are
downloaded using the webapi into `inst/cohorts' and
`inst/sql/sql\_server'. The mapping functions for each measurement
covariate are saved as rds files into the `inst/settings' directory. If
you do not want to include measurement based on being in/not in a cohort
covariates set `measurementCohortCovariateSettings' to NULL.

\hypertarget{finalmapping}{%
\subsubsection{finalMapping}\label{finalmapping}}

The `finalMapping' input is a function that maps from the sum of the
points to a predicted risk. For example, if the model is a logistic
regression, the following mapping would be used:

\begin{Shaded}
\begin{Highlighting}[]
\ControlFlowTok{function}\NormalTok{(x)\{}\KeywordTok{return}\NormalTok{(}\DecValTok{1}\OperatorTok{/}\NormalTok{(}\DecValTok{1}\OperatorTok{+}\KeywordTok{exp}\NormalTok{(}\OperatorTok{-}\NormalTok{x)))\}}
\end{Highlighting}
\end{Shaded}

This gets saved as an rds file in the `inst/setting' directory.

\hypertarget{example}{%
\subsubsection{Example}\label{example}}

A complete example code for inserting the pooled cohort question
non-black model into the skeleton package:

\begin{Shaded}
\begin{Highlighting}[]
\KeywordTok{populatePackageModels}\NormalTok{(}\DataTypeTok{modelname =} \StringTok{'pooled_male_non_black'}\NormalTok{,}
                      \DataTypeTok{standardCovariates =} \OtherTok{NULL}\NormalTok{,}
                      \DataTypeTok{cohortCovariateSettings =} \KeywordTok{list}\NormalTok{(}\DataTypeTok{baseUrl =} \StringTok{'https://yourWebAPI'}\NormalTok{,}
                                                     \DataTypeTok{atlasCovariateIds =} \KeywordTok{c}\NormalTok{(}\DecValTok{17720}\NormalTok{,}\DecValTok{18957}\NormalTok{,}\DecValTok{18957}\NormalTok{),}
                                                     \DataTypeTok{atlasCovariateNames =} \KeywordTok{c}\NormalTok{(}\StringTok{'diabetes'}\NormalTok{, }\StringTok{'smoking'}\NormalTok{,}\StringTok{'smoking'}\NormalTok{),}
                                                     \DataTypeTok{analysisIds =} \KeywordTok{c}\NormalTok{(}\DecValTok{456}\NormalTok{,}\DecValTok{456}\NormalTok{,}\DecValTok{455}\NormalTok{),}
                                                     \DataTypeTok{startDays =} \KeywordTok{c}\NormalTok{(}\OperatorTok{-}\DecValTok{9999}\NormalTok{,}\OperatorTok{-}\DecValTok{730}\NormalTok{,}\OperatorTok{-}\DecValTok{730}\NormalTok{),}
                                                     \DataTypeTok{endDays =} \KeywordTok{c}\NormalTok{(}\OperatorTok{-}\DecValTok{1}\NormalTok{,}\DecValTok{60}\NormalTok{,}\DecValTok{60}\NormalTok{),}
                                                     \DataTypeTok{points =} \KeywordTok{c}\NormalTok{(}\FloatTok{0.658}\NormalTok{,}\FloatTok{7.837}\NormalTok{,}\OperatorTok{-}\FloatTok{1.795}\NormalTok{),}
                                                     \DataTypeTok{count =} \KeywordTok{rep}\NormalTok{(F, }\DecValTok{3}\NormalTok{),}
                                                     \DataTypeTok{ageInteraction =} \KeywordTok{c}\NormalTok{(F,F,F), }
                                                     \DataTypeTok{lnAgeInteraction =} \KeywordTok{c}\NormalTok{(F,F,T)) ,}
                      
                      \DataTypeTok{measurementCovariateSettings =} \KeywordTok{list}\NormalTok{(}\DataTypeTok{names =} \KeywordTok{c}\NormalTok{(}\StringTok{'Total_Cholesterol_mgdL'}\NormalTok{, }\StringTok{'Total_Cholesterol_mgdL'}\NormalTok{,}
                                                                    \StringTok{'HDL-C_mgdL'}\NormalTok{, }\StringTok{'HDL-C_mgdL'}
\NormalTok{                                                                    ),}
                                                          \DataTypeTok{conceptSets =} \KeywordTok{list}\NormalTok{(}\KeywordTok{c}\NormalTok{(}\DecValTok{2212267}\NormalTok{,}\DecValTok{3015232}\NormalTok{,}\DecValTok{3019900}\NormalTok{,}\DecValTok{3027114}\NormalTok{,}\DecValTok{4008265}\NormalTok{,}\DecValTok{4190897}\NormalTok{,}\DecValTok{4198448}\NormalTok{,}\DecValTok{4260765}\NormalTok{,}\DecValTok{37393449}\NormalTok{,}\DecValTok{37397989}\NormalTok{,}\DecValTok{40484105}\NormalTok{,}\DecValTok{44791053}\NormalTok{,}\DecValTok{44809580}\NormalTok{),}
                                                                             \KeywordTok{c}\NormalTok{(}\DecValTok{2212267}\NormalTok{,}\DecValTok{3015232}\NormalTok{,}\DecValTok{3019900}\NormalTok{,}\DecValTok{3027114}\NormalTok{,}\DecValTok{4008265}\NormalTok{,}\DecValTok{4190897}\NormalTok{,}\DecValTok{4198448}\NormalTok{,}\DecValTok{4260765}\NormalTok{,}\DecValTok{37393449}\NormalTok{,}\DecValTok{37397989}\NormalTok{,}\DecValTok{40484105}\NormalTok{,}\DecValTok{44791053}\NormalTok{,}\DecValTok{44809580}\NormalTok{),}
                                                                             \KeywordTok{c}\NormalTok{(}\DecValTok{2212449}\NormalTok{,}\DecValTok{3003767}\NormalTok{,}\DecValTok{3007070}\NormalTok{,}\DecValTok{3007676}\NormalTok{,}\DecValTok{3011884}\NormalTok{,}\DecValTok{3013473}\NormalTok{,}\DecValTok{3015204}\NormalTok{,}\DecValTok{3016945}\NormalTok{,}\DecValTok{3022449}\NormalTok{,}\DecValTok{3023602}\NormalTok{,}\DecValTok{3023752}\NormalTok{,}\DecValTok{3024401}\NormalTok{,}\DecValTok{3030792}\NormalTok{,}\DecValTok{3032771}\NormalTok{,}\DecValTok{3033638}\NormalTok{,}\DecValTok{3034482}\NormalTok{,}\DecValTok{3040815}\NormalTok{,}\DecValTok{3053286}\NormalTok{,}\DecValTok{4005504}\NormalTok{,}\DecValTok{4008127}\NormalTok{,}\DecValTok{4011133}\NormalTok{,}\DecValTok{4019543}\NormalTok{,}\DecValTok{4041557}\NormalTok{,}\DecValTok{4041720}\NormalTok{,}\DecValTok{4042059}\NormalTok{,}\DecValTok{4042081}\NormalTok{,}\DecValTok{4055665}\NormalTok{,}\DecValTok{4076704}\NormalTok{,}\DecValTok{4101713}\NormalTok{,}\DecValTok{4195503}\NormalTok{,}\DecValTok{4198116}\NormalTok{,}\DecValTok{37208659}\NormalTok{,}\DecValTok{37208661}\NormalTok{,}\DecValTok{37392562}\NormalTok{,}\DecValTok{37392938}\NormalTok{,}\DecValTok{37394092}\NormalTok{,}\DecValTok{37394229}\NormalTok{,}\DecValTok{37394230}\NormalTok{,}\DecValTok{37398699}\NormalTok{,}\DecValTok{40757503}\NormalTok{,}\DecValTok{40765014}\NormalTok{,}\DecValTok{44789188}\NormalTok{,}\DecValTok{45768617}\NormalTok{,}\DecValTok{45768651}\NormalTok{,}\DecValTok{45768652}\NormalTok{,}\DecValTok{45768653}\NormalTok{,}\DecValTok{45768654}\NormalTok{,}\DecValTok{45771001}\NormalTok{,}\DecValTok{45772902}\NormalTok{),}
                                                                             \KeywordTok{c}\NormalTok{(}\DecValTok{2212449}\NormalTok{,}\DecValTok{3003767}\NormalTok{,}\DecValTok{3007070}\NormalTok{,}\DecValTok{3007676}\NormalTok{,}\DecValTok{3011884}\NormalTok{,}\DecValTok{3013473}\NormalTok{,}\DecValTok{3015204}\NormalTok{,}\DecValTok{3016945}\NormalTok{,}\DecValTok{3022449}\NormalTok{,}\DecValTok{3023602}\NormalTok{,}\DecValTok{3023752}\NormalTok{,}\DecValTok{3024401}\NormalTok{,}\DecValTok{3030792}\NormalTok{,}\DecValTok{3032771}\NormalTok{,}\DecValTok{3033638}\NormalTok{,}\DecValTok{3034482}\NormalTok{,}\DecValTok{3040815}\NormalTok{,}\DecValTok{3053286}\NormalTok{,}\DecValTok{4005504}\NormalTok{,}\DecValTok{4008127}\NormalTok{,}\DecValTok{4011133}\NormalTok{,}\DecValTok{4019543}\NormalTok{,}\DecValTok{4041557}\NormalTok{,}\DecValTok{4041720}\NormalTok{,}\DecValTok{4042059}\NormalTok{,}\DecValTok{4042081}\NormalTok{,}\DecValTok{4055665}\NormalTok{,}\DecValTok{4076704}\NormalTok{,}\DecValTok{4101713}\NormalTok{,}\DecValTok{4195503}\NormalTok{,}\DecValTok{4198116}\NormalTok{,}\DecValTok{37208659}\NormalTok{,}\DecValTok{37208661}\NormalTok{,}\DecValTok{37392562}\NormalTok{,}\DecValTok{37392938}\NormalTok{,}\DecValTok{37394092}\NormalTok{,}\DecValTok{37394229}\NormalTok{,}\DecValTok{37394230}\NormalTok{,}\DecValTok{37398699}\NormalTok{,}\DecValTok{40757503}\NormalTok{,}\DecValTok{40765014}\NormalTok{,}\DecValTok{44789188}\NormalTok{,}\DecValTok{45768617}\NormalTok{,}\DecValTok{45768651}\NormalTok{,}\DecValTok{45768652}\NormalTok{,}\DecValTok{45768653}\NormalTok{,}\DecValTok{45768654}\NormalTok{,}\DecValTok{45771001}\NormalTok{,}\DecValTok{45772902}\NormalTok{)}
\NormalTok{                                                          ),}
                                                          \DataTypeTok{startDays =} \KeywordTok{c}\NormalTok{(}\OperatorTok{-}\DecValTok{365}\NormalTok{, }\DecValTok{-365}\NormalTok{,}\OperatorTok{-}\DecValTok{365}\NormalTok{, }\DecValTok{-365}\NormalTok{),}
                                                          \DataTypeTok{endDays =} \KeywordTok{c}\NormalTok{(}\DecValTok{60}\NormalTok{,}\DecValTok{60}\NormalTok{,}\DecValTok{60}\NormalTok{,}\DecValTok{60}\NormalTok{),}
                                                          \DataTypeTok{scaleMaps=} \KeywordTok{list}\NormalTok{(}\ControlFlowTok{function}\NormalTok{(x)\{ x =}\StringTok{ }\NormalTok{dplyr}\OperatorTok{::}\KeywordTok{mutate}\NormalTok{(x, }\DataTypeTok{rawValue =}\NormalTok{ dplyr}\OperatorTok{::}\KeywordTok{case_when}\NormalTok{(unitConceptId }\OperatorTok{==}\StringTok{ }\DecValTok{8753} \OperatorTok{~}\StringTok{ }\NormalTok{rawValue}\OperatorTok{*}\FloatTok{38.6}\NormalTok{, unitConceptId }\OperatorTok\StringTok{ }\KeywordTok{c}\NormalTok{(}\DecValTok{8840}\NormalTok{,}\DecValTok{8954}\NormalTok{,}\DecValTok{9028}\NormalTok{ ) }\OperatorTok{~}\StringTok{ }\NormalTok{rawValue, }\OtherTok{TRUE} \OperatorTok{~}\StringTok{ }\DecValTok{0}\NormalTok{)); x=}\StringTok{ }\NormalTok{dplyr}\OperatorTok{::}\KeywordTok{filter}\NormalTok{(x, rawValue }\OperatorTok{>=}\StringTok{ }\DecValTok{80} \OperatorTok{&}\StringTok{ }\NormalTok{rawValue }\OperatorTok{<=}\StringTok{ }\DecValTok{500}\NormalTok{ ); x =}\StringTok{ }\NormalTok{dplyr}\OperatorTok{::}\KeywordTok{mutate}\NormalTok{(x,}\DataTypeTok{valueAsNumber =} \KeywordTok{log}\NormalTok{(rawValue)); }\KeywordTok{return}\NormalTok{(x)\},}
                                                                          \ControlFlowTok{function}\NormalTok{(x)\{ x =}\StringTok{ }\NormalTok{dplyr}\OperatorTok{::}\KeywordTok{mutate}\NormalTok{(x, }\DataTypeTok{rawValue =}\NormalTok{ dplyr}\OperatorTok{::}\KeywordTok{case_when}\NormalTok{(unitConceptId }\OperatorTok{==}\StringTok{ }\DecValTok{8753} \OperatorTok{~}\StringTok{ }\NormalTok{rawValue}\OperatorTok{*}\FloatTok{38.6}\NormalTok{, unitConceptId }\OperatorTok\StringTok{ }\KeywordTok{c}\NormalTok{(}\DecValTok{8840}\NormalTok{,}\DecValTok{8954}\NormalTok{,}\DecValTok{9028}\NormalTok{ ) }\OperatorTok{~}\StringTok{ }\NormalTok{rawValue, }\OtherTok{TRUE} \OperatorTok{~}\StringTok{ }\DecValTok{0}\NormalTok{)); x=}\StringTok{ }\NormalTok{dplyr}\OperatorTok{::}\KeywordTok{filter}\NormalTok{(x, rawValue }\OperatorTok{>=}\StringTok{ }\DecValTok{80} \OperatorTok{&}\StringTok{ }\NormalTok{rawValue }\OperatorTok{<=}\StringTok{ }\DecValTok{500}\NormalTok{ ); x =}\StringTok{ }\NormalTok{dplyr}\OperatorTok{::}\KeywordTok{mutate}\NormalTok{(x,}\DataTypeTok{valueAsNumber =} \KeywordTok{log}\NormalTok{(rawValue)}\OperatorTok{*}\KeywordTok{log}\NormalTok{(age)); }\KeywordTok{return}\NormalTok{(x)\},}
                                                                          \ControlFlowTok{function}\NormalTok{(x)\{ x =}\StringTok{ }\NormalTok{dplyr}\OperatorTok{::}\KeywordTok{mutate}\NormalTok{(x, }\DataTypeTok{rawValue =}\NormalTok{ dplyr}\OperatorTok{::}\KeywordTok{case_when}\NormalTok{(unitConceptId }\OperatorTok{==}\StringTok{ }\DecValTok{8753} \OperatorTok{~}\StringTok{ }\NormalTok{rawValue}\OperatorTok{*}\FloatTok{38.6}\NormalTok{, unitConceptId }\OperatorTok\StringTok{ }\KeywordTok{c}\NormalTok{(}\DecValTok{8840}\NormalTok{,}\DecValTok{8954}\NormalTok{,}\DecValTok{9028}\NormalTok{ ) }\OperatorTok{~}\StringTok{ }\NormalTok{rawValue, }\OtherTok{TRUE} \OperatorTok{~}\StringTok{ }\DecValTok{0}\NormalTok{)); x=}\StringTok{ }\NormalTok{dplyr}\OperatorTok{::}\KeywordTok{filter}\NormalTok{(x, rawValue }\OperatorTok{>=}\StringTok{ }\DecValTok{10} \OperatorTok{&}\StringTok{ }\NormalTok{rawValue }\OperatorTok{<=}\StringTok{ }\DecValTok{130}\NormalTok{ ); x =}\StringTok{ }\NormalTok{dplyr}\OperatorTok{::}\KeywordTok{mutate}\NormalTok{(x,}\DataTypeTok{valueAsNumber =} \KeywordTok{log}\NormalTok{(rawValue)); }\KeywordTok{return}\NormalTok{(x)\},}
                                                                          \ControlFlowTok{function}\NormalTok{(x)\{ x =}\StringTok{ }\NormalTok{dplyr}\OperatorTok{::}\KeywordTok{mutate}\NormalTok{(x, }\DataTypeTok{rawValue =}\NormalTok{ dplyr}\OperatorTok{::}\KeywordTok{case_when}\NormalTok{(unitConceptId }\OperatorTok{==}\StringTok{ }\DecValTok{8753} \OperatorTok{~}\StringTok{ }\NormalTok{rawValue}\OperatorTok{*}\FloatTok{38.6}\NormalTok{, unitConceptId }\OperatorTok\StringTok{ }\KeywordTok{c}\NormalTok{(}\DecValTok{8840}\NormalTok{,}\DecValTok{8954}\NormalTok{,}\DecValTok{9028}\NormalTok{ ) }\OperatorTok{~}\StringTok{ }\NormalTok{rawValue, }\OtherTok{TRUE} \OperatorTok{~}\StringTok{ }\DecValTok{0}\NormalTok{)); x=}\StringTok{ }\NormalTok{dplyr}\OperatorTok{::}\KeywordTok{filter}\NormalTok{(x, rawValue }\OperatorTok{>=}\StringTok{ }\DecValTok{10} \OperatorTok{&}\StringTok{ }\NormalTok{rawValue }\OperatorTok{<=}\StringTok{ }\DecValTok{130}\NormalTok{ ); x =}\StringTok{ }\NormalTok{dplyr}\OperatorTok{::}\KeywordTok{mutate}\NormalTok{(x,}\DataTypeTok{valueAsNumber =} \KeywordTok{log}\NormalTok{(rawValue)}\OperatorTok{*}\KeywordTok{log}\NormalTok{(age)); }\KeywordTok{return}\NormalTok{(x)\}), }
                                                          \DataTypeTok{points =} \KeywordTok{c}\NormalTok{(}\FloatTok{11.853}\NormalTok{,}\OperatorTok{-}\FloatTok{2.664}\NormalTok{,}\OperatorTok{-}\FloatTok{7.990}\NormalTok{,}\FloatTok{1.769}\NormalTok{),}
                                                          \DataTypeTok{aggregateMethods =} \KeywordTok{c}\NormalTok{(}\StringTok{'recent'}\NormalTok{,}\StringTok{'recent'}\NormalTok{,}\StringTok{'recent'}\NormalTok{, }\StringTok{'recent'}\NormalTok{),}
                                                          \DataTypeTok{imputationValues =} \KeywordTok{c}\NormalTok{(}\DecValTok{150}\NormalTok{,}\DecValTok{150}\NormalTok{,}\DecValTok{50}\NormalTok{,}\DecValTok{50}\NormalTok{),}
                                                          \DataTypeTok{ageInteractions =} \KeywordTok{c}\NormalTok{(F,F,F,F),}
                                                          \DataTypeTok{lnAgeInteractions =} \KeywordTok{c}\NormalTok{(F,T,F,T),}
                                                          \DataTypeTok{lnValues =} \KeywordTok{c}\NormalTok{(T,T,T,T),}
                                                          \DataTypeTok{measurementIds =} \KeywordTok{c}\NormalTok{(}\DecValTok{1}\NormalTok{,}\DecValTok{2}\NormalTok{,}\DecValTok{3}\NormalTok{,}\DecValTok{4}\NormalTok{), }
                                                          \DataTypeTok{analysisIds =} \KeywordTok{c}\NormalTok{(}\DecValTok{457}\NormalTok{,}\DecValTok{457}\NormalTok{,}\DecValTok{457}\NormalTok{,}\DecValTok{457}\NormalTok{)}
                                                          
                                                          
\NormalTok{                      ),}
                      
                      \DataTypeTok{ageCovariateSettings =} \KeywordTok{list}\NormalTok{(}\DataTypeTok{names =} \KeywordTok{c}\NormalTok{(}\StringTok{'log(age)'}\NormalTok{),}
                                                  \DataTypeTok{ageMaps =} \KeywordTok{list}\NormalTok{(}\ControlFlowTok{function}\NormalTok{(x)\{}\KeywordTok{return}\NormalTok{(}\KeywordTok{log}\NormalTok{(x))\}),}
                                                  \DataTypeTok{ageIds =} \DecValTok{1}\NormalTok{,}
                                                  \DataTypeTok{analysisIds =} \KeywordTok{c}\NormalTok{(}\DecValTok{458}\NormalTok{),}
                                                  \DataTypeTok{points =} \KeywordTok{c}\NormalTok{(}\FloatTok{12.344}\NormalTok{)}
                                                  
\NormalTok{                      ),}
                      
                      \DataTypeTok{measurementCohortCovariateSettings =} \KeywordTok{list}\NormalTok{(}\DataTypeTok{names =} \KeywordTok{c}\NormalTok{(}\StringTok{'treated_Systolic_BP_mm_Hg'}\NormalTok{,}\StringTok{'untreated_Systolic_BP_mm_Hg'}\NormalTok{),}
                                                                \DataTypeTok{atlasCovariateIds =} \KeywordTok{c}\NormalTok{(}\StringTok{'18946'}\NormalTok{,}\StringTok{'18946'}\NormalTok{),}
                                                                \DataTypeTok{types =} \KeywordTok{c}\NormalTok{(}\StringTok{'in'}\NormalTok{, }\StringTok{'out'}\NormalTok{),}
                                                                \DataTypeTok{conceptSets =} \KeywordTok{list}\NormalTok{(}\KeywordTok{c}\NormalTok{(}\DecValTok{3004249}\NormalTok{,}\DecValTok{3009395}\NormalTok{,}\DecValTok{3018586}\NormalTok{,}\DecValTok{3028737}\NormalTok{,}\DecValTok{3035856}\NormalTok{,}\DecValTok{4152194}\NormalTok{,}\DecValTok{4153323}\NormalTok{,}\DecValTok{4161413}\NormalTok{,}\DecValTok{4197167}\NormalTok{,}\DecValTok{4217013}\NormalTok{,}\DecValTok{4232915}\NormalTok{,}\DecValTok{4248525}\NormalTok{,}\DecValTok{4292062}\NormalTok{,}\DecValTok{21492239}\NormalTok{,}\DecValTok{37396683}\NormalTok{,}\DecValTok{44789315}\NormalTok{,}\DecValTok{44806887}\NormalTok{,}\DecValTok{45769778}\NormalTok{),}
                                                                                   \KeywordTok{c}\NormalTok{(}\DecValTok{3004249}\NormalTok{,}\DecValTok{3009395}\NormalTok{,}\DecValTok{3018586}\NormalTok{,}\DecValTok{3028737}\NormalTok{,}\DecValTok{3035856}\NormalTok{,}\DecValTok{4152194}\NormalTok{,}\DecValTok{4153323}\NormalTok{,}\DecValTok{4161413}\NormalTok{,}\DecValTok{4197167}\NormalTok{,}\DecValTok{4217013}\NormalTok{,}\DecValTok{4232915}\NormalTok{,}\DecValTok{4248525}\NormalTok{,}\DecValTok{4292062}\NormalTok{,}\DecValTok{21492239}\NormalTok{,}\DecValTok{37396683}\NormalTok{,}\DecValTok{44789315}\NormalTok{,}\DecValTok{44806887}\NormalTok{,}\DecValTok{45769778}\NormalTok{)}
                                                                                   
\NormalTok{                                                                ),}
                                                                \DataTypeTok{startDays =} \KeywordTok{c}\NormalTok{(}\OperatorTok{-}\DecValTok{365}\NormalTok{,}\OperatorTok{-}\DecValTok{365}\NormalTok{),}
                                                                \DataTypeTok{endDays =} \KeywordTok{c}\NormalTok{(}\DecValTok{60}\NormalTok{,}\DecValTok{60}\NormalTok{),}
                                                                \DataTypeTok{scaleMaps=} \KeywordTok{list}\NormalTok{(}\ControlFlowTok{function}\NormalTok{(x)\{ x =}\StringTok{ }\NormalTok{dplyr}\OperatorTok{::}\KeywordTok{filter}\NormalTok{(x, rawValue }\OperatorTok{>=}\StringTok{ }\DecValTok{50} \OperatorTok{&}\StringTok{ }\NormalTok{rawValue }\OperatorTok{<=}\StringTok{ }\DecValTok{250}\NormalTok{ ); }\KeywordTok{return}\NormalTok{(x)\},}
                                                                                \ControlFlowTok{function}\NormalTok{(x)\{ x =}\StringTok{ }\NormalTok{dplyr}\OperatorTok{::}\KeywordTok{filter}\NormalTok{(x, rawValue }\OperatorTok{>=}\StringTok{ }\DecValTok{50} \OperatorTok{&}\StringTok{ }\NormalTok{rawValue }\OperatorTok{<=}\StringTok{ }\DecValTok{250}\NormalTok{ ); }\KeywordTok{return}\NormalTok{(x)\}}
\NormalTok{                                                                ),}
                                                                \DataTypeTok{points =} \KeywordTok{c}\NormalTok{(}\FloatTok{1.797}\NormalTok{,}\FloatTok{1.764}\NormalTok{),}
                                                                \DataTypeTok{aggregateMethods =} \KeywordTok{c}\NormalTok{(}\StringTok{'recent'}\NormalTok{,}\StringTok{'recent'}\NormalTok{),}
                                                                \DataTypeTok{imputationValues =} \KeywordTok{c}\NormalTok{(}\DecValTok{120}\NormalTok{,}\DecValTok{120}\NormalTok{),}
                                                                \DataTypeTok{ageInteractions =} \KeywordTok{c}\NormalTok{(F,F),}
                                                                \DataTypeTok{lnAgeInteractions =} \KeywordTok{c}\NormalTok{(F,F),}
                                                                \DataTypeTok{lnValues =} \KeywordTok{c}\NormalTok{(T,T),}
                                                                \DataTypeTok{measurementIds =} \KeywordTok{c}\NormalTok{(}\DecValTok{1}\NormalTok{,}\DecValTok{2}\NormalTok{), }
                                                                \DataTypeTok{analysisIds =} \KeywordTok{c}\NormalTok{(}\DecValTok{459}\NormalTok{,}\DecValTok{459}\NormalTok{),}
                                                                \DataTypeTok{baseUrl =} \StringTok{'https://yourWebAPI'}
                                                                
                                                                
\NormalTok{                      ),}
                      
                      \DataTypeTok{finalMapping =} \ControlFlowTok{function}\NormalTok{(x)\{ }\DecValTok{1}\OperatorTok{-}\StringTok{ }\FloatTok{0.9144}\OperatorTok{^}\KeywordTok{exp}\NormalTok{(x}\FloatTok{-61.18}\NormalTok{)\}}
\NormalTok{)}
\end{Highlighting}
\end{Shaded}

\hypertarget{step-5-build-the-study-package}{%
\subsection{Step 5: Build the study
package}\label{step-5-build-the-study-package}}

Aftering adding the settings into the package, you now need to build the
package. Use the standard process (in R studio press the `Build' tab in
the top right corner and then select the `Install and Restart' button)
to build the study package so an R library is created.

\end{document}
